\documentclass{article}
\usepackage[utf8]{inputenc}

\begin{document}
\section{Executive Summary}
The objective of this project is for our team to recreate \textit{Atari's} 1981 hit arcade game \textit{Frogger} in Java. This project falls under \textbf{Category C - Animation Application or Game} on the \textit{CPSC 233 Project Proposal Page}. This project will be completed before the end of the 2018 summer semester (within 6 weeks). This project proposal contains the background of the original 1981 Frogger game, an overview of our team's re-creation, and the objectives, and outcome of the project.

\section{Introduction}
\subsection{Background - 1981 Frogger}
The objective of the game is for the player to guide five frog avatars individually to a designated space across the screen while avoiding traffic vehicles. The frog is only allowed to move in four directions - up, down, left, and right. Players are able to control the frog via a four-key directional pad; each key push in a certain direction will cause the frog to move one spot in said direction. The player's failure to avoid said traffic vehicles while guiding one of five frogs to the designated space across the screen will cause the player to lose one of their limited lives. Losing all lives before guiding all five frogs to their designated spaces will cause the player to lose the game.

\subsection{Overview of Our Team's Re-Creation}
To differentiate from the original arcade game, our re-creation will not contain five frog avatars. Instead, our re-creation will contain only one frog avatar and three custom maps. The player will be required to guide the frog to the designated space on all three maps to win the game. Upon reaching the designated space of one map, the game will continue onto the next map until the last map has been completed. Again, the player must avoid traffic vehicles and has a limited number of lives. These changes make it possible for the team, with limited programming experience, to complete the project within the 6 week time frame and allows team members to be creative with designing the custom maps. The re-creation will contain a score-based system that will keep track of user high-scores.

\section{Objectives}
Objectives adapted from material prepared by \textit{Dr. Sohaib Bajwa} for \textit{CPSC 233 S18}:
\begin{itemize}
    \item A graphical user interface (GUI) and a text-based interface.
    \item Keeping track of dynamic/static items in the game.
    \item Collision detection (e.g. death upon hitting traffic or advancement in the game after reaching the designated space).
    \item Creating a text-based version of our re-creation of the game based on user input instead of a timer.
    \item Getting user input via the keyboard.
    \item Completing all assigned demos and milestones.
\end{itemize}

\section{Project Outcome}
A fully functional re-creation of the hit 1981 game \textit{Frogger} by \textit{Atari} produced within six weeks. The re-creation will be based on the original game, but will contain numerous changes including, but not limited to, one frog instead of five, three maps that are accessed by progressing through the game instead of only one map, and a high-score based tracking system.
\end{document}
